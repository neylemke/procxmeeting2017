
\documentclass[twoside]{article}
\usepackage[affil-it]{authblk}
\usepackage{lipsum} % Package to generate dummy text throughout this template
\usepackage{eurosym}
\usepackage[sc]{mathpazo} % Use the Palatino font
\usepackage[T1]{fontenc} % Use 8-bit encoding that has 256 glyphs
\usepackage[utf8]{inputenc}
\linespread{1.05} % Line spacing-Palatino needs more space between lines
\usepackage{microtype} % Slightly tweak font spacing for aesthetics

\usepackage[hmarginratio=1:1,top=32mm,columnsep=20pt]{geometry} % Document margins
\usepackage{multicol} % Used for the two-column layout of the document
\usepackage[hang,small,labelfont=bf,up,textfont=it,up]{caption} % Custom captions under//above floats in tables or figures
\usepackage{booktabs} % Horizontal rules in tables
\usepackage{float} % Required for tables and figures in the multi-column environment-they need to be placed in specific locations with the[H] (e.g. \begin{table}[H])
\usepackage{hyperref} % For hyperlinks in the PDF

\usepackage{lettrine} % The lettrine is the first enlarged letter at the beginning of the text
\usepackage{paralist} % Used for the compactitem environment which makes bullet points with less space between them

\usepackage{abstract} % Allows abstract customization
\renewcommand{\abstractnamefont}{\normalfont\bfseries} 
%\renewcommand{\abstracttextfont}{\normalfont\small\itshape} % Set the abstract itself to small italic text

\usepackage{titlesec} % Allows customization of titles
\renewcommand\thesection{\Roman{section}} % Roman numerals for the sections
\renewcommand\thesubsection{\Roman{subsection}} % Roman numerals for subsections
\titleformat{\section}[block]{\large\scshape\centering}{\thesection.}{1em}{} % Change the look of the section titles
\titleformat{\subsection}[block]{\large}{\thesubsection.}{1em}{} % Change the look of the section titles

\usepackage{fancyhdr} % Headers and footers
\pagestyle{fancy} % All pages have headers and footers
\fancyhead{} % Blank out the default header
\fancyfoot{} % Blank out the default footer
\fancyhead[C]{X-meeting $\bullet$ November 2017 $\bullet$ S\~ao Pedro} % Custom header text
\fancyfoot[RO,LE]{} % Custom footer text

%----------------------------------------------------------------------------------------
% TITLE SECTION
%----------------------------------------------------------------------------------------

\title{\vspace{-15mm}\fontsize{24pt}{10pt}\selectfont\textbf{GeNICE: A Novel Framework for Gene Network Inference by Clustering, Exhaustive Search, and Multivariate Analysis}} % Article title

\author{Ricardo de Souza Jacomini$^1$, David Correa Martins Jr$^2$, Felipe Leno da Silva$^1$, Anna Helena Reali Costa$^1$}

\affil{1 ESCOLA POLIT\'ECNICA DA USP\\ 2 UFABC\\ }
\vspace{-5mm}
\date{}

%----------------------------------------------------------------------------------------

\begin{document}

\maketitle % Insert title

\thispagestyle{fancy} % All pages have headers and footers

%----------------------------------------------------------------------------------------
% ABSTRACT
%----------------------------------------------------------------------------------------

\begin{abstract}
Gene network (GN) inference from temporal gene expression data is a crucial and challenging problem in systems biology. Expression data sets usually consist of dozens of temporal samples, while networks consist of thousands of genes, thus rendering many inference methods unfeasible in practice. To improve the scalability of GN inference methods, we propose a novel framework called GeNICE, based on probabilistic GNs; the main novelty is the introduction of a clustering procedure to group genes with related expression profiles and to provide an approximate solution with reduced computational complexity. We use the defined clusters to perform an exhaustive search to retrieve the best predictor gene subsets for each target gene, according to multivariate criterion functions. GeNICE greatly reduces the search space because predictor candidates are restricted to one gene per cluster. Finally, a multivariate analysis is performed for each defined predictor subset to retrieve minimal subsets and to simplify the network. In our experiments with in silico generated data sets, GeNICE achieved substantial computational time reduction when compared to solutions without the clustering step, while preserving the gene expression prediction accuracy even when the number of clusters is small (about 50) relative to the number of genes (order of thousands). For a Plasmodium falciparum microarray data set, the prediction accuracy achieved by GeNICE was roughly 97\%, while the respective topologies involving glycolytic and apicoplast seed genes had a very large intramodularity, very small interconnection between modules, and some module hub genes, reflecting small-world and scale-free topological properties, as expected.

Funding: CNPq, FAPESP
\end{abstract}
\end{document}