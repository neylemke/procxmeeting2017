
  \documentclass[twoside]{article}
  \usepackage[affil-it]{authblk}
  \usepackage{lipsum} % Package to generate dummy text throughout this template
  \usepackage{eurosym}
  \usepackage[sc]{mathpazo} % Use the Palatino font
  \usepackage[T1]{fontenc} % Use 8-bit encoding that has 256 glyphs
  \usepackage[utf8]{inputenc}
  \linespread{1.05} % Line spacing-Palatino needs more space between lines
  \usepackage{microtype} % Slightly tweak font spacing for aesthetics\[IndentingNewLine]
  \usepackage[hmarginratio=1:1,top=32mm,columnsep=20pt]{geometry} % Document margins
  \usepackage{multicol} % Used for the two-column layout of the document
  \usepackage[hang,small,labelfont=bf,up,textfont=it,up]{caption} % Custom captions under//above floats in tables or figures
  \usepackage{booktabs} % Horizontal rules in tables
  \usepackage{float} % Required for tables and figures in the multi-column environment-they need to be placed in specific locations with the[H] (e.g. \begin{table}[H])
  \usepackage{hyperref} % For hyperlinks in the PDF
  \usepackage{lettrine} % The lettrine is the first enlarged letter at the beginning of the text
  \usepackage{paralist} % Used for the compactitem environment which makes bullet points with less space between them
  \usepackage{abstract} % Allows abstract customization
  \renewcommand{\abstractnamefont}{\normalfont\bfseries} 
  %\renewcommand{\abstracttextfont}{\normalfont\small\itshape} % Set the abstract itself to small italic text\[IndentingNewLine]
  \usepackage{titlesec} % Allows customization of titles
  \renewcommand\thesection{\Roman{section}} % Roman numerals for the sections
  \renewcommand\thesubsection{\Roman{subsection}} % Roman numerals for subsections
  \titleformat{\section}[block]{\large\scshape\centering}{\thesection.}{1em}{} % Change the look of the section titles
  \titleformat{\subsection}[block]{\large}{\thesubsection.}{1em}{} % Change the look of the section titles
  \usepackage{fancyhdr} % Headers and footers
  \pagestyle{fancy} % All pages have headers and footers
  \fancyhead{} % Blank out the default header
  \fancyfoot{} % Blank out the default footer
  \fancyhead[C]{X-meeting $\bullet$ November 2017 $\bullet$ S\~ao Pedro} % Custom header text
  \fancyfoot[RO,LE]{} % Custom footer text
  %----------------------------------------------------------------------------------------
  % TITLE SECTION
  %---------------------------------------------------------------------------------------- 
 
 \title{\vspace{-15mm}\fontsize{24pt}{10pt}\selectfont\textbf{ Metalloproteinases diversity in the venom gland of Peruvian spider Loxosceles laeta revealed by transcriptome analysis }} % Article title
  
  
  \author{ Raissa Medina Santos$^{1}$, Clara Guerra Duarte$^{2}$, Priscilla Alves de Aquino$^{1}$, Anderson Oliveira do Carmo$^{1}$, César Bonilla$^{3}$, Evanguedes Kalapothakis$^{1}$, Carlos Chavez-Olortegui$^{1}$, }
  
  \affil{ 1 Universidade Federal de Minas Gerais

2 Fundação Ezequiel Dias

3 Instituto Nacional de Salud

 }
  \vspace{-5mm}
  \date{}
  
  %---------------------------------------------------------------------------------------- 
  
  \begin{document}
  
  
  \maketitle % Insert title
  
  
  \thispagestyle{fancy} % All pages have headers and footers
  %----------------------------------------------------------------------------------------  
  % ABSTRACT
  
  %----------------------------------------------------------------------------------------  
  
  \begin{abstract}
  Envenomation caused by spiders from Loxosceles genus (brown spiders) is a worldwide public health problem. Loxosceles their venom is composed of several toxins responsible for dermonecrotic, hemorrhagic and edema effects. In Peru, L. laeta is considered  the most medical relevant species. A family of metalloproteases, also named astacin-like proteins, was described in Loxosceles venom with great importance for hemostatic disorders in natural or experimental envenomations. A new generation sequencing library of venom extracted from the Peruvian spider, L. laeta, was constructed for the first time using the TruSeq ™ RNA Sample Prep Kit v3 Set A (Illumina) kit and the sequencing was performed on the MiSeq by the paired-end technique for identification of molecular diversity of metalloproteases toxins. In this work, we describe some of the identified metalloproteases enzymes with a high degree of identity (over 50\%) with molecules from other Loxosceles spp spiders. Results obtained in this work represent the first landscape of components of a Peruvian spider venom gland, revealing the complexity of molecules expressed in this tissue, with great potential for future uses in medical and evolutionary studies.
  
  Funding: FAPEMIG, CNPq, CAPES \\ 
  \end{abstract}
  \end{document} 