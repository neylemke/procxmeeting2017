
  \documentclass[twoside]{article}
  \usepackage[affil-it]{authblk}
  \usepackage{lipsum} % Package to generate dummy text throughout this template
  \usepackage{eurosym}
  \usepackage[sc]{mathpazo} % Use the Palatino font
  \usepackage[T1]{fontenc} % Use 8-bit encoding that has 256 glyphs
  \usepackage[utf8]{inputenc}
  \linespread{1.05} % Line spacing-Palatino needs more space between lines
  \usepackage{microtype} % Slightly tweak font spacing for aesthetics\[IndentingNewLine]
  \usepackage[hmarginratio=1:1,top=32mm,columnsep=20pt]{geometry} % Document margins
  \usepackage{multicol} % Used for the two-column layout of the document
  \usepackage[hang,small,labelfont=bf,up,textfont=it,up]{caption} % Custom captions under//above floats in tables or figures
  \usepackage{booktabs} % Horizontal rules in tables
  \usepackage{float} % Required for tables and figures in the multi-column environment-they need to be placed in specific locations with the[H] (e.g. \begin{table}[H])
  \usepackage{hyperref} % For hyperlinks in the PDF
  \usepackage{lettrine} % The lettrine is the first enlarged letter at the beginning of the text
  \usepackage{paralist} % Used for the compactitem environment which makes bullet points with less space between them
  \usepackage{abstract} % Allows abstract customization
  \renewcommand{\abstractnamefont}{\normalfont\bfseries} 
  %\renewcommand{\abstracttextfont}{\normalfont\small\itshape} % Set the abstract itself to small italic text\[IndentingNewLine]
  \usepackage{titlesec} % Allows customization of titles
  \renewcommand\thesection{\Roman{section}} % Roman numerals for the sections
  \renewcommand\thesubsection{\Roman{subsection}} % Roman numerals for subsections
  \titleformat{\section}[block]{\large\scshape\centering}{\thesection.}{1em}{} % Change the look of the section titles
  \titleformat{\subsection}[block]{\large}{\thesubsection.}{1em}{} % Change the look of the section titles
  \usepackage{fancyhdr} % Headers and footers
  \pagestyle{fancy} % All pages have headers and footers
  \fancyhead{} % Blank out the default header
  \fancyfoot{} % Blank out the default footer
  \fancyhead[C]{X-meeting $\bullet$ November 2017 $\bullet$ S\~ao Pedro} % Custom header text
  \fancyfoot[RO,LE]{} % Custom footer text
  %----------------------------------------------------------------------------------------
  % TITLE SECTION
  %---------------------------------------------------------------------------------------- 
 
 \title{\vspace{-15mm}\fontsize{24pt}{10pt}\selectfont\textbf{ Best Practices for Bioinformatics Pipelines for Molecular-Barcoded Targeted Sequencing }} % Article title
  
  
  \author{ Marcel Caraciolo$^{1}$, Wilder Barbosa Galvao$^{1}$, George de Vasconcelos Carvalho Neto$^{1}$, Rodrigo Bertollo$^{1}$, Joao Bosco Oliveira$^{1}$, }
  
  \affil{ 1 Genomika

 }
  \vspace{-5mm}
  \date{}
  
  %---------------------------------------------------------------------------------------- 
  
  \begin{document}
  
  
  \maketitle % Insert title
  
  
  \thispagestyle{fancy} % All pages have headers and footers
  %----------------------------------------------------------------------------------------  
  % ABSTRACT
  
  %----------------------------------------------------------------------------------------  
  
  \begin{abstract}
  In cancer research the detection of mutations is critical, for tumor samples and blood samples mutations may be present in very low fractions of DNA molecules. By using molecular barcoding technology, more than reduce the impact of enrichment, the sequencing errors can be eliminated by tagging each input molecule with an unique molecular identifier (UMI). In contrast to sample barcoding, molecular barcoding assigns a unique sequence not just to all the molecules from a certain sample, but to all molecules being amplified and sequenced. Despite the difference it is common to have both sample barcodes and molecular barcodes in the same sequencing reads. Recent works on this approach show outstanding performance in targeted high-throughput sequencing, being the most promising approach for the accurate identification of rare variants in complex DNA samples, and has application in several areas such as detecting DNA mutations at very low allele fractions with high accuracy for cancer samples and reducing sequencing artifacts occurrences. However,  at the sample preparation, the residual PCR errors might be introduced at first PCR cycles and during UMI tag attachment, which decrease the accuracy of variant calling.  In order to perform the variant detection on those input data, a different approach is required for bioinformatics pipelines that handles the caveats of UMI-based analysis. By using specific algorithms and softwares, the pipeline is designed to obtain high-fidelity mutation profiles and call ultra-rare variants.  In this poster we present the best practices and strategies for handling the UMI-tagged data, by showing the steps and related software tools to the audience when building the variant calling pipeline
  
  Funding: Genomika \\ 
  \end{abstract}
  \end{document} 