
\documentclass[twoside]{article}
\usepackage[affil-it]{authblk}
\usepackage{lipsum} % Package to generate dummy text throughout this template
\usepackage{eurosym}
\usepackage[sc]{mathpazo} % Use the Palatino font
\usepackage[T1]{fontenc} % Use 8-bit encoding that has 256 glyphs
\usepackage[utf8]{inputenc}
\linespread{1.05} % Line spacing-Palatino needs more space between lines
\usepackage{microtype} % Slightly tweak font spacing for aesthetics

\usepackage[hmarginratio=1:1,top=32mm,columnsep=20pt]{geometry} % Document margins
\usepackage{multicol} % Used for the two-column layout of the document
\usepackage[hang,small,labelfont=bf,up,textfont=it,up]{caption} % Custom captions under//above floats in tables or figures
\usepackage{booktabs} % Horizontal rules in tables
\usepackage{float} % Required for tables and figures in the multi-column environment-they need to be placed in specific locations with the[H] (e.g. \begin{table}[H])
\usepackage{hyperref} % For hyperlinks in the PDF

\usepackage{lettrine} % The lettrine is the first enlarged letter at the beginning of the text
\usepackage{paralist} % Used for the compactitem environment which makes bullet points with less space between them

\usepackage{abstract} % Allows abstract customization
\renewcommand{\abstractnamefont}{\normalfont\bfseries} 
%\renewcommand{\abstracttextfont}{\normalfont\small\itshape} % Set the abstract itself to small italic text

\usepackage{titlesec} % Allows customization of titles
\renewcommand\thesection{\Roman{section}} % Roman numerals for the sections
\renewcommand\thesubsection{\Roman{subsection}} % Roman numerals for subsections
\titleformat{\section}[block]{\large\scshape\centering}{\thesection.}{1em}{} % Change the look of the section titles
\titleformat{\subsection}[block]{\large}{\thesubsection.}{1em}{} % Change the look of the section titles

\usepackage{fancyhdr} % Headers and footers
\pagestyle{fancy} % All pages have headers and footers
\fancyhead{} % Blank out the default header
\fancyfoot{} % Blank out the default footer
\fancyhead[C]{X-meeting $\bullet$ November 2017 $\bullet$ S\~ao Pedro} % Custom header text
\fancyfoot[RO,LE]{} % Custom footer text

%----------------------------------------------------------------------------------------
% TITLE SECTION
%----------------------------------------------------------------------------------------

\title{\vspace{-15mm}\fontsize{24pt}{10pt}\selectfont\textbf{R package development to analyze the cancer genome atlas data: a study case based on hypoxia induced factor- a3 isoforms}} % Article title

\author{F\'abio Malta de S\'a Patroni$^1$, Douglas Adamoski$^2$, Marcelo Falsarella Carazzolle$^3$, Sandra Martha Gomes Dias$^4$}

\affil{1 GRADUATE PROGRAM IN GENETICS AND MOLECULAR BIOLOGY, INSTITUTE OF BIOLOGY UNICAMP, CAMPINAS\\ 2 GRADUATE PROGRAM IN GENETICS AND MOLECULAR BIOLOGY, INSTITUTE OF BIOLOGY UNICAMP\\ 3 BIOLOGY INSTITUTE - UNICAMP, NATIONAL CENTER FOR HIGH PERFORMANCE COMPUTING/UNICAMP\\ 4 BRAZILIAN NATIONAL CENTER FOR RESEARCH IN ENERGY AND MATERIALS, BRAZILIAN BIOSCIENCES NATIONAL LABORATORY\\ }
\vspace{-5mm}
\date{}

%----------------------------------------------------------------------------------------

\begin{document}

\maketitle % Insert title

\thispagestyle{fancy} % All pages have headers and footers

%----------------------------------------------------------------------------------------
% ABSTRACT
%----------------------------------------------------------------------------------------

\begin{abstract}
A great magnitude of information on multi-resource omics data is being created and made freely available through the project The Cancer Genome Atlas (TCGA). Although the amount of data rises every year, data mining tools are not following at the same pace. The efficient use of this information, also called the ``big data'', has the potential to unveil new observations and mechanisms that can impact on cancer treatment. TCGA data are stored at the GDC Data Portal and GDC Legacy Archive, both of which hosted by the US Nacional Cancer Institute (NCI). The totality of the data comprehends genomic, transcriptomic, proteomic and metilome information from 33 types of cancer. Given the complexity and extension of the available data, new analytical tools are necessary to automate and facilitate the data mining process. R programming language is being widely used for dealing with ``big data''. This work has two main goals: First, to create an R package aiming at to download, organize, analyze and report TCGA data; second, apply the package to identify potential downstream targets of the transcriptional factor HIF-3a isoforms. HIF regulates the expression of genes as a response to hypoxia and is an important player on the tumor metabolism adaptation process. Our R package, GDCRtools (version: 0.0.9) was developed and used to analyze the HIF3a2 isoform in four tumor types: Ovarian serous cystadenocarcinoma [OV], Testicular Germ Cell Tumors [TGCT], Uterine Carcinosarcoma [UCS] and Stomach adenocarcinoma [STAD].Tumors were divided among higher and lower HIF3a2 expression, and differential gene expression determined. Gene Ontology and Reactome was employed for pathway enrichment analysis and revealed enriched terms related with extracellular matrix organization, blood vessel development and GPCR downstream signaling, potentially linking this isoform with these processes.

Funding: This work was supported by grants from S\~ao Paulo Research Foundation (2015/26059-7)
\end{abstract}
\end{document}