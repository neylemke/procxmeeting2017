
\documentclass[twoside]{article}
\usepackage[affil-it]{authblk}
\usepackage{lipsum} % Package to generate dummy text throughout this template
\usepackage{eurosym}
\usepackage[sc]{mathpazo} % Use the Palatino font
\usepackage[T1]{fontenc} % Use 8-bit encoding that has 256 glyphs
\usepackage[utf8]{inputenc}
\linespread{1.05} % Line spacing-Palatino needs more space between lines
\usepackage{microtype} % Slightly tweak font spacing for aesthetics

\usepackage[hmarginratio=1:1,top=32mm,columnsep=20pt]{geometry} % Document margins
\usepackage{multicol} % Used for the two-column layout of the document
\usepackage[hang,small,labelfont=bf,up,textfont=it,up]{caption} % Custom captions under//above floats in tables or figures
\usepackage{booktabs} % Horizontal rules in tables
\usepackage{float} % Required for tables and figures in the multi-column environment-they need to be placed in specific locations with the[H] (e.g. \begin{table}[H])
\usepackage{hyperref} % For hyperlinks in the PDF

\usepackage{lettrine} % The lettrine is the first enlarged letter at the beginning of the text
\usepackage{paralist} % Used for the compactitem environment which makes bullet points with less space between them

\usepackage{abstract} % Allows abstract customization
\renewcommand{\abstractnamefont}{\normalfont\bfseries} 
%\renewcommand{\abstracttextfont}{\normalfont\small\itshape} % Set the abstract itself to small italic text

\usepackage{titlesec} % Allows customization of titles
\renewcommand\thesection{\Roman{section}} % Roman numerals for the sections
\renewcommand\thesubsection{\Roman{subsection}} % Roman numerals for subsections
\titleformat{\section}[block]{\large\scshape\centering}{\thesection.}{1em}{} % Change the look of the section titles
\titleformat{\subsection}[block]{\large}{\thesubsection.}{1em}{} % Change the look of the section titles

\usepackage{fancyhdr} % Headers and footers
\pagestyle{fancy} % All pages have headers and footers
\fancyhead{} % Blank out the default header
\fancyfoot{} % Blank out the default footer
\fancyhead[C]{X-meeting $\bullet$ November 2017 $\bullet$ S\~ao Pedro} % Custom header text
\fancyfoot[RO,LE]{} % Custom footer text

%----------------------------------------------------------------------------------------
% TITLE SECTION
%----------------------------------------------------------------------------------------

\title{\vspace{-15mm}\fontsize{24pt}{10pt}\selectfont\textbf{Transcriptional evaluation of induced pluripotent cells from patients with Cockayne syndrome after induction of DNA damage triggered by oxidative stress}} % Article title

\author{Maira Rodrigues de Camargo Neves$^1$, Livia Luz Souza Nascimento$^1$, Alexandre Teixeira Vessoni$^2$, Carlos Frederico Martins Menck$^1$}

\affil{1 DEPARTMENT OF MICROBIOLOGY, INSTITUTE OF BIOMEDICAL SCIENCES, USP\\ 2 DEPARTMENT OF MEDICINE, WASHINGTON UNIVERSITY IN SAINT LOUIS\\ }
\vspace{-5mm}
\date{}

%----------------------------------------------------------------------------------------

\begin{document}

\maketitle % Insert title

\thispagestyle{fancy} % All pages have headers and footers

%----------------------------------------------------------------------------------------
% ABSTRACT
%----------------------------------------------------------------------------------------

\begin{abstract}
Cockayne Syndrome (CS) is characterized by symptoms related to premature ageing with
severe involvement of the central nervous system. The molecular basis of the disease is related
to deficiency in the transcription-coupled repair (TCR), mainly with mutations in the ERCC8 and
ERCC6 genes (coding for CSA and CSB proteins, respectively). The phenotype of CS cells is
presented as high sensitivity to ultraviolet (UV) light, causing DNA damage, which in turn
prevents transcription recovery after irradiation. They are also more susceptible to DNA
damage caused by oxidative stress, which maybe responsible for endogenous DNA lesions.
Although it has been proposed that the CS transcriptional pattern following DNA lesions might
be responsible for the cellular and clinical phenotype of patients, this pattern has not been
investigated yet for stem cells. In the present work, we are investigating the transcription
pattern through RNAseq in CS induced pluripotent stem cells (iPSCs) following DNA damage by
oxidative stress. Preliminary tests for cell survival determination allowed the standardization of
a Potassium bromate (KBrO3) concentration for DNA damage challenge experiments.
Experiments were conducted on a wild type cell strain (F9048), and on a CSB mutant
(GM10903, Coriell), both reprogrammed to iPS. Libraries were prepared for RNAseq with
mRNA from both cell strains, extracted 24 h after KBrO3 and mock treatments. Sequencing
was conducted on an Illumina NextSeq, with paired-end reads. The run yielded 637 million
clusters, with an average of 52 million paired-end reads per sample. Data analysis was
performed with the HISAT2-StringTie-Ballgown protocol (Tuxedo 2), against Ensembl GRCh38
genome. RSeQC was used for quality control and determination of median transcript integrity
number (medTIN), and distribution of reads along each transcript to exclude library
preparation bias. CS cells presented 109 differentially expressed genes, all observed exclusively
on these cells following DNA damage challenge. Interestingly, only one gene (VLDLR-AS1) was
identified as differentially expressed in wild type cells under the same treatment, suggesting
CS cells are more sensitive to transcriptional variation after oxidative stress. An enrichment of
GO terms for the regulation pathway of insulin growth factor (IGF) was found among the
differentially expressed genes on CS cells, but not on wild type cells, corroborating previous
findings. Furthermore, over half of the differentially expressed genes on CS are associated with
the GO biological process ``response to stimulus'', mainly ``response to stress''. Five
differentially expressed genes were classified as GO neuron projection regeneration (ULK1,
SPP1, APOE, ADM, JUN), and possibly contribute to the severe nervous system involvement in
the patients phenotype.

Funding: CAPES and FAPESP
\end{abstract}
\end{document}