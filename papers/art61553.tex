
\documentclass[twoside]{article}
\usepackage[affil-it]{authblk}
\usepackage{lipsum} % Package to generate dummy text throughout this template
\usepackage{eurosym}
\usepackage[sc]{mathpazo} % Use the Palatino font
\usepackage[T1]{fontenc} % Use 8-bit encoding that has 256 glyphs
\usepackage[utf8]{inputenc}
\linespread{1.05} % Line spacing-Palatino needs more space between lines
\usepackage{microtype} % Slightly tweak font spacing for aesthetics

\usepackage[hmarginratio=1:1,top=32mm,columnsep=20pt]{geometry} % Document margins
\usepackage{multicol} % Used for the two-column layout of the document
\usepackage[hang,small,labelfont=bf,up,textfont=it,up]{caption} % Custom captions under//above floats in tables or figures
\usepackage{booktabs} % Horizontal rules in tables
\usepackage{float} % Required for tables and figures in the multi-column environment-they need to be placed in specific locations with the[H] (e.g. \begin{table}[H])
\usepackage{hyperref} % For hyperlinks in the PDF

\usepackage{lettrine} % The lettrine is the first enlarged letter at the beginning of the text
\usepackage{paralist} % Used for the compactitem environment which makes bullet points with less space between them

\usepackage{abstract} % Allows abstract customization
\renewcommand{\abstractnamefont}{\normalfont\bfseries} 
%\renewcommand{\abstracttextfont}{\normalfont\small\itshape} % Set the abstract itself to small italic text

\usepackage{titlesec} % Allows customization of titles
\renewcommand\thesection{\Roman{section}} % Roman numerals for the sections
\renewcommand\thesubsection{\Roman{subsection}} % Roman numerals for subsections
\titleformat{\section}[block]{\large\scshape\centering}{\thesection.}{1em}{} % Change the look of the section titles
\titleformat{\subsection}[block]{\large}{\thesubsection.}{1em}{} % Change the look of the section titles

\usepackage{fancyhdr} % Headers and footers
\pagestyle{fancy} % All pages have headers and footers
\fancyhead{} % Blank out the default header
\fancyfoot{} % Blank out the default footer
\fancyhead[C]{X-meeting $\bullet$ November 2017 $\bullet$ S\~ao Pedro} % Custom header text
\fancyfoot[RO,LE]{} % Custom footer text

%----------------------------------------------------------------------------------------
% TITLE SECTION
%----------------------------------------------------------------------------------------

\title{\vspace{-15mm}\fontsize{24pt}{10pt}\selectfont\textbf{PacBio assembly of a Plasmodium knowlesi genome sequence with Hi-C correction and manual annotation of the SICAvar gene family}} % Article title

\author{Juliana Assis$^1$}

\affil{1 UFMG\\ }
\vspace{-5mm}
\date{}

%----------------------------------------------------------------------------------------

\begin{document}

\maketitle % Insert title

\thispagestyle{fancy} % All pages have headers and footers

%----------------------------------------------------------------------------------------
% ABSTRACT
%----------------------------------------------------------------------------------------

\begin{abstract}
Plasmodium knowlesi has risen in importance as a zoonotic parasite that has been causing regular episodes of malaria throughout South East Asia. The P. knowlesi genome sequence generated in 2008 highlighted and confirmed many simi- larities and differences in Plasmodium species, including a global view of several multigene families, such as the large SICAvar multigene family encoding the variant antigens known as the schizont-infected cell agglutination proteins. However, repetitive DNA sequences are the bane of any genome project, and this and other Plasmodium genome projects have not been immune to the gaps, rearrangements and other pitfalls created by these genomic features. Today, long-read PacBio and chromatin conformation technologies are overcoming such obstacles. Here, based on the use of these technolo- gies, we present a highly refined de novo P. knowlesi genome sequence of the Pk1(A+) clone. This sequence and annotation, referred to as the 'MaHPIC Pk genome sequence', includes manual annotation of the SICAvar gene family with 136 full- length members categorized as type I or II. This sequence provides a framework that will permit a better understanding of the SICAvar repertoire, selective pressures acting on this gene family and mechanisms of antigenic variation in this species and other pathogens.

Funding: Federal funds from the National Institute of Allergy and Infectious Diseases; National Institutes of Health, Department of Health and Human Services (Contract No.HHSN272201200031C) and the National Center for Research Resources (ORIP/OD P51OD011132). This study was also financially supported by the National Institutes of Health (R01 AI06775-01) to KGLR, the University of California, Riverside (NIFA-Hatch-225935) to KGLR and Institute Leadership Funds from La Jolla Institute for Allergy and Immun
\end{abstract}
\end{document}