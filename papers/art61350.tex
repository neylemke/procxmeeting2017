
\documentclass[twoside]{article}
\usepackage[affil-it]{authblk}
\usepackage{lipsum} % Package to generate dummy text throughout this template
\usepackage{eurosym}
\usepackage[sc]{mathpazo} % Use the Palatino font
\usepackage[T1]{fontenc} % Use 8-bit encoding that has 256 glyphs
\usepackage[utf8]{inputenc}
\linespread{1.05} % Line spacing-Palatino needs more space between lines
\usepackage{microtype} % Slightly tweak font spacing for aesthetics

\usepackage[hmarginratio=1:1,top=32mm,columnsep=20pt]{geometry} % Document margins
\usepackage{multicol} % Used for the two-column layout of the document
\usepackage[hang,small,labelfont=bf,up,textfont=it,up]{caption} % Custom captions under//above floats in tables or figures
\usepackage{booktabs} % Horizontal rules in tables
\usepackage{float} % Required for tables and figures in the multi-column environment-they need to be placed in specific locations with the[H] (e.g. \begin{table}[H])
\usepackage{hyperref} % For hyperlinks in the PDF

\usepackage{lettrine} % The lettrine is the first enlarged letter at the beginning of the text
\usepackage{paralist} % Used for the compactitem environment which makes bullet points with less space between them

\usepackage{abstract} % Allows abstract customization
\renewcommand{\abstractnamefont}{\normalfont\bfseries} 
%\renewcommand{\abstracttextfont}{\normalfont\small\itshape} % Set the abstract itself to small italic text

\usepackage{titlesec} % Allows customization of titles
\renewcommand\thesection{\Roman{section}} % Roman numerals for the sections
\renewcommand\thesubsection{\Roman{subsection}} % Roman numerals for subsections
\titleformat{\section}[block]{\large\scshape\centering}{\thesection.}{1em}{} % Change the look of the section titles
\titleformat{\subsection}[block]{\large}{\thesubsection.}{1em}{} % Change the look of the section titles

\usepackage{fancyhdr} % Headers and footers
\pagestyle{fancy} % All pages have headers and footers
\fancyhead{} % Blank out the default header
\fancyfoot{} % Blank out the default footer
\fancyhead[C]{X-meeting $\bullet$ November 2017 $\bullet$ S\~ao Pedro} % Custom header text
\fancyfoot[RO,LE]{} % Custom footer text

%----------------------------------------------------------------------------------------
% TITLE SECTION
%----------------------------------------------------------------------------------------

\title{\vspace{-15mm}\fontsize{24pt}{10pt}\selectfont\textbf{SnoRNA and piRNA expression levels modified by tobacco use in women with lung adenocarcinoma}} % Article title

\author{Natasha Jorge$^1$, Gabriel Wajnberg$^2$, Carlos Gil Ferreira$^3$, Ben\'{\i}lton Carvalho$^4$, Fabio Passetti$^1$}

\affil{1 FIOCRUZ - IOC\\ 2 FIOCRUZ-IOC\\ 3 D'OR INSTITUTE FOR RESERACH AND EDUCATION, RIO DE JANEIRO, RJ, BRAZIL\\ 4 DEPARTMENT OF STATISTICS, STATE UNICAMP, CAMPINAS, SP, BRAZIL\\ }
\vspace{-5mm}
\date{}

%----------------------------------------------------------------------------------------

\begin{document}

\maketitle % Insert title

\thispagestyle{fancy} % All pages have headers and footers

%----------------------------------------------------------------------------------------
% ABSTRACT
%----------------------------------------------------------------------------------------

\begin{abstract}
Lung cancer is one of the most frequent types of cancer and a major cause of death related to cancer worldwide. The incidence of this disease is directly related with tobacco usage, however not every smoker develops lung cancer and a third of the female lung cancer Asian patients are non-smokers. Small non coding RNAs such as, small nucleolar RNAs (snoRNAs) and piwi interacting RNAs (piRNA), have been used as normalization parameter and are considered putative biomarkers for the prediction of treatment outcome. We utilized  data of small RNA high throughput sequencing of matched  control tissue adjacent to the tumor and lung adenocarcinoma tumor samples from 6 female non-smokers and 5 female smokers to identify snoRNAs and piRNAs which expression is altered (DE) by tobacco usage and those which expression remains the same (CN). Here, we report, for the first time, 49 snoRNAs and piRNAs DE between the normal samples of smokers and non-smokers and 55 between the tumor samples. Out of the DE genes, 28 were found in common in both analyses presenting the same trend, thus reinforcing that these alterations were due to tobacco usage. One example is SNORD66, which was found up-regulated in smokers samples and was already suggested as a biomarker for lung cancer. A total of 16 snoRNAs or piRNAs were found CN between smokers and non-smokers control and tumor samples, one example is U43, which is frequently used as normalization parameter in molecular biology experiments. Our findings indicate that both tumor samples and adjacent control tissue samples from smokers and non-smokers have distinct snoRNAs and piRNAs expression profiles, which can be used to better understand the effects of tobacco on the cell metabolism and to differentiate the samples, and identifies other snoRNAs whose expression does not change neither in tumors nor control samples by tobacco use, thus can be used for standardization and normalization of other experiments with similar samples.

Financial Support: CAPES, CNPq, FAPERJ, and FAPESP

Funding: CAPES, CNPq, FAPERJ, FAPESP
\end{abstract}
\end{document}