
\documentclass[twoside]{article}
\usepackage[affil-it]{authblk}
\usepackage{lipsum} % Package to generate dummy text throughout this template
\usepackage{eurosym}
\usepackage[sc]{mathpazo} % Use the Palatino font
\usepackage[T1]{fontenc} % Use 8-bit encoding that has 256 glyphs
\usepackage[utf8]{inputenc}
\linespread{1.05} % Line spacing-Palatino needs more space between lines
\usepackage{microtype} % Slightly tweak font spacing for aesthetics

\usepackage[hmarginratio=1:1,top=32mm,columnsep=20pt]{geometry} % Document margins
\usepackage{multicol} % Used for the two-column layout of the document
\usepackage[hang,small,labelfont=bf,up,textfont=it,up]{caption} % Custom captions under//above floats in tables or figures
\usepackage{booktabs} % Horizontal rules in tables
\usepackage{float} % Required for tables and figures in the multi-column environment-they need to be placed in specific locations with the[H] (e.g. \begin{table}[H])
\usepackage{hyperref} % For hyperlinks in the PDF

\usepackage{lettrine} % The lettrine is the first enlarged letter at the beginning of the text
\usepackage{paralist} % Used for the compactitem environment which makes bullet points with less space between them

\usepackage{abstract} % Allows abstract customization
\renewcommand{\abstractnamefont}{\normalfont\bfseries} 
%\renewcommand{\abstracttextfont}{\normalfont\small\itshape} % Set the abstract itself to small italic text

\usepackage{titlesec} % Allows customization of titles
\renewcommand\thesection{\Roman{section}} % Roman numerals for the sections
\renewcommand\thesubsection{\Roman{subsection}} % Roman numerals for subsections
\titleformat{\section}[block]{\large\scshape\centering}{\thesection.}{1em}{} % Change the look of the section titles
\titleformat{\subsection}[block]{\large}{\thesubsection.}{1em}{} % Change the look of the section titles

\usepackage{fancyhdr} % Headers and footers
\pagestyle{fancy} % All pages have headers and footers
\fancyhead{} % Blank out the default header
\fancyfoot{} % Blank out the default footer
\fancyhead[C]{X-meeting $\bullet$ November 2017 $\bullet$ S\~ao Pedro} % Custom header text
\fancyfoot[RO,LE]{} % Custom footer text

%----------------------------------------------------------------------------------------
% TITLE SECTION
%----------------------------------------------------------------------------------------

\title{\vspace{-15mm}\fontsize{24pt}{10pt}\selectfont\textbf{Genome-wide identification of novel miRNAs in cnidarian genomes}} % Article title

\author{Tamires Caixeta Alves$^1$, Laurence Rodrigues do Amaral$^1$, Matheus de Souza Gomes$^1$}

\affil{1 UFU\\ }
\vspace{-5mm}
\date{}

%----------------------------------------------------------------------------------------

\begin{document}

\maketitle % Insert title

\thispagestyle{fancy} % All pages have headers and footers

%----------------------------------------------------------------------------------------
% ABSTRACT
%----------------------------------------------------------------------------------------

\begin{abstract}
Cnidarian is a phylum of the kingdom Animalia of great ecological and economic importance, due to its peculiar characteristics like, for example, its high adaptive capacity to the most diverse environments. The mechanisms that mediate changes in gene expression in response to stress remain unknown and there is a need to look at the regulatory mechanisms that control the dynamic expression of genes in the face of environmental challenges. Recent studies have shown the importance of gene regulation involving small RNAs, their processing system and their performance at the cellular level. MicroRNAs are considered one of the most important noncoding small RNAs silencing the mRNAs controlling their gene expression. Computational methods have been applied to identify and characterize putative miRNAs, their precursor and mRNA target genes. MiRNAs in plants and animals are already well elucidated. However, there is an evolutionary gap that needs to be addressed about the emergence of miRNAs. The most commonly accepted hypothesis is that of convergent evolution. We believed that studies of cnidarian can better explain it, either to affirm or deny the hypothesis. To obtain further clarification on the cnidarians, we searched for precursor, mature and miRNA targets in three species of cnidarians: Acropora digitifera, Exaiptasia pallida and Hydra vulgaris. An optimized algorithm, with stringent filters according to the conserved characteristics of miRNAs, was used to search for precursors. For search of mature miRNAs in the sequences of identified precursors, we aligned sequences deposited in miRBase version 21 (http://www.mirbase.org/). The RNAfold program was used to predict the secondary structure of the miRNA precursor and the RNAalifold and ClustalX 2.1 programs were used to generate alignments of the precursors and their orthologs. Phylogenetic analysis was performed with the aid of MEGA5.2 program. We identified 77 miRNA precursors, 77 mature, within 66 families in Acropora digitifera; 22 miRNA precursors, 22 mature, within 21 families in Exaiptasia pallida; 12 miRNA precursors, 12 mature, within 11 families in Hydra vulgaris. miRNAs plants and bilateral animals were found in our study. This suggested a common ancestor between cnidarians and plants. Phylogenetic analysis showed that our results corroborated the tree of life. Thus, our results expanded the study of small RNAs involved in cnidarians, and provided an alternative explanation on the evolution of miRNAs.

Funding: FAPEMIG, CNPq, UFU and CAPES
\end{abstract}
\end{document}