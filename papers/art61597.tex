
  \documentclass[twoside]{article}
  \usepackage[affil-it]{authblk}
  \usepackage{lipsum} % Package to generate dummy text throughout this template
  \usepackage{eurosym}
  \usepackage[sc]{mathpazo} % Use the Palatino font
  \usepackage[T1]{fontenc} % Use 8-bit encoding that has 256 glyphs
  \usepackage[utf8]{inputenc}
  \linespread{1.05} % Line spacing-Palatino needs more space between lines
  \usepackage{microtype} % Slightly tweak font spacing for aesthetics\[IndentingNewLine]
  \usepackage[hmarginratio=1:1,top=32mm,columnsep=20pt]{geometry} % Document margins
  \usepackage{multicol} % Used for the two-column layout of the document
  \usepackage[hang,small,labelfont=bf,up,textfont=it,up]{caption} % Custom captions under//above floats in tables or figures
  \usepackage{booktabs} % Horizontal rules in tables
  \usepackage{float} % Required for tables and figures in the multi-column environment-they need to be placed in specific locations with the[H] (e.g. \begin{table}[H])
  \usepackage{hyperref} % For hyperlinks in the PDF
  \usepackage{lettrine} % The lettrine is the first enlarged letter at the beginning of the text
  \usepackage{paralist} % Used for the compactitem environment which makes bullet points with less space between them
  \usepackage{abstract} % Allows abstract customization
  \renewcommand{\abstractnamefont}{\normalfont\bfseries} 
  %\renewcommand{\abstracttextfont}{\normalfont\small\itshape} % Set the abstract itself to small italic text\[IndentingNewLine]
  \usepackage{titlesec} % Allows customization of titles
  \renewcommand\thesection{\Roman{section}} % Roman numerals for the sections
  \renewcommand\thesubsection{\Roman{subsection}} % Roman numerals for subsections
  \titleformat{\section}[block]{\large\scshape\centering}{\thesection.}{1em}{} % Change the look of the section titles
  \titleformat{\subsection}[block]{\large}{\thesubsection.}{1em}{} % Change the look of the section titles
  \usepackage{fancyhdr} % Headers and footers
  \pagestyle{fancy} % All pages have headers and footers
  \fancyhead{} % Blank out the default header
  \fancyfoot{} % Blank out the default footer
  \fancyhead[C]{X-meeting $\bullet$ November 2017 $\bullet$ S\~ao Pedro} % Custom header text
  \fancyfoot[RO,LE]{} % Custom footer text
  %----------------------------------------------------------------------------------------
  % TITLE SECTION
  %---------------------------------------------------------------------------------------- 
 
 \title{\vspace{-15mm}\fontsize{24pt}{10pt}\selectfont\textbf{ Analysis of metagenomic data from howler monkeys feces }} % Article title
  
  
  \author{ Italo Sudre Pereira$^{1}$, Raquel Riyuzo de Almeida Franco$^{2}$, Layla Martins$^{1}$, Julio Oliveira$^{3}$, João Carlos Setubal$^{2}$, Aline Maria da Silva$^{2}$, }
  
  \affil{ 1 Universidade de São Paulo

2 USP

3 Universidade Federal de São Paulo

 }
  \vspace{-5mm}
  \date{}
  
  %---------------------------------------------------------------------------------------- 
  
  \begin{document}
  
  
  \maketitle % Insert title
  
  
  \thispagestyle{fancy} % All pages have headers and footers
  %----------------------------------------------------------------------------------------  
  % ABSTRACT
  
  %----------------------------------------------------------------------------------------  
  
  \begin{abstract}
  There is an increasing use of metagenomic approaches to characterize microbial communities in several environments regarding their structure, function and composition, in particular to access the vast amount of uncultured microorganisms, enabling the understanding of the biological functions that such organisms play in these environments. Here we describe results of a project aiming to sample and analyze the fecal microbiota of captive and non-captive howler monkeys in the S\~ao Paulo Zoo. A previous study from our group using 16S rRNA amplicon sequencing has demonstrated differences in the microbiota profile between captive and non-captive individuals. In this project, we performed shotgun sequencing of total DNA obtained from thirteen samples from captive and six from non-captive monkeys, the samples were collected in different seasons. The diet of those groups are different and we have well detailed information about captive diet. Preliminary results show differences between the two groups in terms of taxonomic groups as well as in the function profile. The taxonomic results shows that Bacteroides, Prevotella and Parabacteroides are the most abundant genera in non-captive monkeys and Prevotella, Bacteroides and Clostridium are the most abundant genera in captive monkeys. The taxonomic and functional profiles results suggests that the non-captive monkeys are more susceptible to season changes and captive ones have a more homogeneous microbiota over the year.
  
  Funding: FAPESP, CAPES, CNPq \\ 
  \end{abstract}
  \end{document} 