
\documentclass[twoside]{article}
\usepackage[affil-it]{authblk}
\usepackage{lipsum} % Package to generate dummy text throughout this template
\usepackage{eurosym}
\usepackage[sc]{mathpazo} % Use the Palatino font
\usepackage[T1]{fontenc} % Use 8-bit encoding that has 256 glyphs
\usepackage[utf8]{inputenc}
\linespread{1.05} % Line spacing-Palatino needs more space between lines
\usepackage{microtype} % Slightly tweak font spacing for aesthetics

\usepackage[hmarginratio=1:1,top=32mm,columnsep=20pt]{geometry} % Document margins
\usepackage{multicol} % Used for the two-column layout of the document
\usepackage[hang,small,labelfont=bf,up,textfont=it,up]{caption} % Custom captions under//above floats in tables or figures
\usepackage{booktabs} % Horizontal rules in tables
\usepackage{float} % Required for tables and figures in the multi-column environment-they need to be placed in specific locations with the[H] (e.g. \begin{table}[H])
\usepackage{hyperref} % For hyperlinks in the PDF

\usepackage{lettrine} % The lettrine is the first enlarged letter at the beginning of the text
\usepackage{paralist} % Used for the compactitem environment which makes bullet points with less space between them

\usepackage{abstract} % Allows abstract customization
\renewcommand{\abstractnamefont}{\normalfont\bfseries} 
%\renewcommand{\abstracttextfont}{\normalfont\small\itshape} % Set the abstract itself to small italic text

\usepackage{titlesec} % Allows customization of titles
\renewcommand\thesection{\Roman{section}} % Roman numerals for the sections
\renewcommand\thesubsection{\Roman{subsection}} % Roman numerals for subsections
\titleformat{\section}[block]{\large\scshape\centering}{\thesection.}{1em}{} % Change the look of the section titles
\titleformat{\subsection}[block]{\large}{\thesubsection.}{1em}{} % Change the look of the section titles

\usepackage{fancyhdr} % Headers and footers
\pagestyle{fancy} % All pages have headers and footers
\fancyhead{} % Blank out the default header
\fancyfoot{} % Blank out the default footer
\fancyhead[C]{X-meeting $\bullet$ November 2017 $\bullet$ S\~ao Pedro} % Custom header text
\fancyfoot[RO,LE]{} % Custom footer text

%----------------------------------------------------------------------------------------
% TITLE SECTION
%----------------------------------------------------------------------------------------

\title{\vspace{-15mm}\fontsize{24pt}{10pt}\selectfont\textbf{Identification of motifs in the promoter region of genes related to the ABA-dependent pathway in sugarcane}} % Article title

\author{Mauro de Medeiros Oliveira$^1$, Alan Durham$^1$, Glaucia Souza Mendes$^1$}

\affil{1 USP\\ }
\vspace{-5mm}
\date{}

%----------------------------------------------------------------------------------------

\begin{document}

\maketitle % Insert title

\thispagestyle{fancy} % All pages have headers and footers

%----------------------------------------------------------------------------------------
% ABSTRACT
%----------------------------------------------------------------------------------------

\begin{abstract}
In general, the promoter region consists of different regulatory elements, such as Transcription
Factor Binding Sites (TFBS), which are responsible for the activation of gene transcription. TFBSs
can be characterized using different experimental processes such as Chip-Seq. However, these
experimental present low reproducibility for non-model organisms. For these organisms the
dominant form of TFBS discovery is computational estimations using techniques such as
expectation maximization (EM). The goal of this work is to characterize the PR of ABA signaling
pathway (ABAsp) genes using an in silico approach. To perform our analyzes, we used expression
data of the sugarcane variety RB83-5486 to select all ABAsp genes differentially expressed in
drought-tolerant plants. The 22 selected genes were mapped on the sugarcane genome SP80-3280
and the regions of 2000 nucleotides usptream from the transcription start site of each gene were
extracted as putative promoter regions. For these regions we tried 3 different motif-finding
approaches: Gibbs Sampling (using GLAM2), position-specific score matrices for previously
characterized motifs in the JASPAR plant databases, and expectation maximization (using MEME).
Only the last approach resulted in consistent results. GLAM2 showed a bias for AT-rich motifs and
none of the results had any TOMTOM match against the JASPAR plant databases. Analysis using
PSSMs did not find any candidates with significant scores. We parametrized MEME to find motifs
from 5 to 15 nucleotides and maximum of 6 different motifs. We only considered motifs with a
TOMTOM match to JASPAR plant databases. In general, 50\% of the sequences presented similar
TFBS architectures, with the bZIP, WRKY and AP2 / ERF classes of TFBSs as the most
representative. Moreover, we distinguished different architectures for up and for down-regulated
genes: in up-regulated genes we found motifs associated to ARR and bHLH TFBS classes, and in
down-regulated genes were found motifs associated tot he HD-Zip, MYB and NAC TFBS classes.
In this scenario it is possible to infer that the drought tolerance may be due to the crossing of
different signaling pathways for water stress. Since two groups of TFBS distinct from the ABAsp
were identified in the promoter region, one associated with up-regulated genes, and one associated
with down-regulated genes, the architecture of the promoter region may be the factor necessary to
activate the drought tolerant character observed in the evaluated plants.

Funding: FAPESP and Capes
\end{abstract}
\end{document}