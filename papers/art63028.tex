
\documentclass[twoside]{article}
\usepackage[affil-it]{authblk}
\usepackage{lipsum} % Package to generate dummy text throughout this template
\usepackage{eurosym}
\usepackage[sc]{mathpazo} % Use the Palatino font
\usepackage[T1]{fontenc} % Use 8-bit encoding that has 256 glyphs
\usepackage[utf8]{inputenc}
\linespread{1.05} % Line spacing-Palatino needs more space between lines
\usepackage{microtype} % Slightly tweak font spacing for aesthetics

\usepackage[hmarginratio=1:1,top=32mm,columnsep=20pt]{geometry} % Document margins
\usepackage{multicol} % Used for the two-column layout of the document
\usepackage[hang,small,labelfont=bf,up,textfont=it,up]{caption} % Custom captions under//above floats in tables or figures
\usepackage{booktabs} % Horizontal rules in tables
\usepackage{float} % Required for tables and figures in the multi-column environment-they need to be placed in specific locations with the[H] (e.g. \begin{table}[H])
\usepackage{hyperref} % For hyperlinks in the PDF

\usepackage{lettrine} % The lettrine is the first enlarged letter at the beginning of the text
\usepackage{paralist} % Used for the compactitem environment which makes bullet points with less space between them

\usepackage{abstract} % Allows abstract customization
\renewcommand{\abstractnamefont}{\normalfont\bfseries} 
%\renewcommand{\abstracttextfont}{\normalfont\small\itshape} % Set the abstract itself to small italic text

\usepackage{titlesec} % Allows customization of titles
\renewcommand\thesection{\Roman{section}} % Roman numerals for the sections
\renewcommand\thesubsection{\Roman{subsection}} % Roman numerals for subsections
\titleformat{\section}[block]{\large\scshape\centering}{\thesection.}{1em}{} % Change the look of the section titles
\titleformat{\subsection}[block]{\large}{\thesubsection.}{1em}{} % Change the look of the section titles

\usepackage{fancyhdr} % Headers and footers
\pagestyle{fancy} % All pages have headers and footers
\fancyhead{} % Blank out the default header
\fancyfoot{} % Blank out the default footer
\fancyhead[C]{X-meeting $\bullet$ November 2017 $\bullet$ S\~ao Pedro} % Custom header text
\fancyfoot[RO,LE]{} % Custom footer text

%----------------------------------------------------------------------------------------
% TITLE SECTION
%----------------------------------------------------------------------------------------

\title{\vspace{-15mm}\fontsize{24pt}{10pt}\selectfont\textbf{Dynamical model of the Ras-mediated AP-1 activation in mouse Y1 adrenocortical tumor cells.}} % Article title

\author{Vincent Noel$^1$, Marcelo S. Reis$^1$, Matheus H.s. Dias$^1$, Cecilia S. Fonseca$^1$, Francisca N.l. Vitorino$^1$, Layra L. Albuquerque$^1$, Fabio Nakano$^2$, Julia P.c. da Cunha$^1$, Junior Barrera$^3$, Hugo A. Armelin$^1$}

\affil{1 INSTITUTO BUTANTAN\\ 2 ESCOLA DE ARTES, CI\^ENCIAS E HUMANIDADES, USP, BRASIL\\ 3 INSTITUTO DE MATEM\'ATICA E ESTAT\'ISTICA, USP\\ }
\vspace{-5mm}
\date{}

%----------------------------------------------------------------------------------------

\begin{document}

\maketitle % Insert title

\thispagestyle{fancy} % All pages have headers and footers

%----------------------------------------------------------------------------------------
% ABSTRACT
%----------------------------------------------------------------------------------------

\begin{abstract}
The K-Ras-driven mouse adrenocortical tumor cell line Y1 displays a surprising association of phenotypic traits, i.e., high basal levels of activated K-Ras in starved cells and induction of cell cycle arrest upon stimulation by FGF2. In addition, ectopic expression of the dominant negative mutant Ras-N17 reduced activated K-Ras basal levels and eliminated cell cycle arrest by FGF2. We are working to uncovered the molecular basis of this unexpected phenomenon by building a dynamical model of the Ras-MAPK signaling pathway in Y1, and it's subsequent activation of the AP-1 complex. This model has been designed using SigNetSim, a web platform for modeling signaling networks being developed by our team. We notably used its functionality of building hierarchical models to produce a modular network, and simplify the reuse of already existing models. We started our K-Ras molecular switch model by reusing a published Ras model. We then added another GEF to reproduce the high basal K-Ras activation observed in Y1, and a Ras dominant negative which expression would reduce the K-Ras basal level. With these modifications, our model was able to reproduce experimental observations from our team. Then, we linked our K-Ras model to a published model of MAPK pathway, by making its activation dependent on Ras activation. With this additional module, we were able to reproduce experimental observations of MAPK activation. We finally added the translocation of MAPK to the nucleus, and its expression of the AP-1 complex. We were able to start building a model to describe the unusual behavior of Y1 cells. Our model reproduces the behavior of K-Ras, MAPK, and AP-1 activation in starved cells, serum stimulated cells, and Serum+FGF2 stimulated cells. We are presently studying the cell cycle activation by AP-1, and the additional stress produced by FGF2 stimulation, to incorporate it in our model and be able to reproduce the cell cycle blockage. We are also planning to improve  the conditions covered by our model to reproduce the observations on FGF-resistant Y1 sublines.

Funding: This work was supported by grants \#12/20186-9, \#13/07467-1, and \#13/24212-7 of the S\~ao Paulo Research Foundation (FAPESP).
\end{abstract}
\end{document}