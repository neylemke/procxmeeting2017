
  \documentclass[twoside]{article}
  \usepackage[affil-it]{authblk}
  \usepackage{lipsum} % Package to generate dummy text throughout this template
  \usepackage{eurosym}
  \usepackage[sc]{mathpazo} % Use the Palatino font
  \usepackage[T1]{fontenc} % Use 8-bit encoding that has 256 glyphs
  \usepackage[utf8]{inputenc}
  \linespread{1.05} % Line spacing-Palatino needs more space between lines
  \usepackage{microtype} % Slightly tweak font spacing for aesthetics\[IndentingNewLine]
  \usepackage[hmarginratio=1:1,top=32mm,columnsep=20pt]{geometry} % Document margins
  \usepackage{multicol} % Used for the two-column layout of the document
  \usepackage[hang,small,labelfont=bf,up,textfont=it,up]{caption} % Custom captions under//above floats in tables or figures
  \usepackage{booktabs} % Horizontal rules in tables
  \usepackage{float} % Required for tables and figures in the multi-column environment-they need to be placed in specific locations with the[H] (e.g. \begin{table}[H])
  \usepackage{hyperref} % For hyperlinks in the PDF
  \usepackage{lettrine} % The lettrine is the first enlarged letter at the beginning of the text
  \usepackage{paralist} % Used for the compactitem environment which makes bullet points with less space between them
  \usepackage{abstract} % Allows abstract customization
  \renewcommand{\abstractnamefont}{\normalfont\bfseries} 
  %\renewcommand{\abstracttextfont}{\normalfont\small\itshape} % Set the abstract itself to small italic text\[IndentingNewLine]
  \usepackage{titlesec} % Allows customization of titles
  \renewcommand\thesection{\Roman{section}} % Roman numerals for the sections
  \renewcommand\thesubsection{\Roman{subsection}} % Roman numerals for subsections
  \titleformat{\section}[block]{\large\scshape\centering}{\thesection.}{1em}{} % Change the look of the section titles
  \titleformat{\subsection}[block]{\large}{\thesubsection.}{1em}{} % Change the look of the section titles
  \usepackage{fancyhdr} % Headers and footers
  \pagestyle{fancy} % All pages have headers and footers
  \fancyhead{} % Blank out the default header
  \fancyfoot{} % Blank out the default footer
  \fancyhead[C]{X-meeting $\bullet$ November 2017 $\bullet$ S\~ao Pedro} % Custom header text
  \fancyfoot[RO,LE]{} % Custom footer text
  %----------------------------------------------------------------------------------------
  % TITLE SECTION
  %---------------------------------------------------------------------------------------- 
 
 \title{\vspace{-15mm}\fontsize{24pt}{10pt}\selectfont\textbf{ Development of an integrated genetic map for a full-sib progeny from crossing between Eucalyptus grandis and Eucalyptus urophylla }} % Article title
  
  
  \author{ Cristiane Hayumi Taniguti$^{1}$, Izabel Christina Gava de Souza$^{2}$, Shinitiro Oda$^{2}$, Leandro de Siqueira$^{2}$, Rodrigo Neves Graça$^{3}$, Thiago Romanos Benatti$^{2}$, José Luiz Stape$^{2}$, Antonio Augusto Franco Garcia$^{1}$, }
  
  \affil{ 1 Universidade de São Paulo

2 Suzano Papel e Celulose

3 Futuragene

 }
  \vspace{-5mm}
  \date{}
  
  %---------------------------------------------------------------------------------------- 
  
  \begin{document}
  
  
  \maketitle % Insert title
  
  
  \thispagestyle{fancy} % All pages have headers and footers
  %----------------------------------------------------------------------------------------  
  % ABSTRACT
  
  %----------------------------------------------------------------------------------------  
  
  \begin{abstract}
  Brazil is among the largest eucalyptus producers in the world. The culture has great importance in Brazilian economy, attending a variety of markers, as the cellulose market. The commercial importance of this crop requires constant improvement of the cultivars. The increasing improvement and availability of the genotyping and phenotyping high-throughput platforms consolidate promising proposals to accelerate eucalyptus breeding programs. The data generated by such platforms are used for genetic understanding of quantitative traits, which are the majority of the commercially-targeted characteristics. Linkage maps are fundamental tools for analyzing these characters. However, it is necessary the development of new strategies for the construction of genetic maps adapted to high-throughput data. These strategies also requires considering particular genetic aspects of eucalyptus species, as their outcrossing breeding system, which makes available only F1 mapping populations. These populations can have greater number of alleles per locus and unknown linkage phases compared to inbred based populations. Furthermore, molecular characteristics obtained by eucalyptus genome sequencing can be useful to linkage map building process. The aim in the present work was to construct an integrated genetic map in a full-sib progeny with 200 individuals, derived from the cross between Eucalyptus grandis and Eucalyptus urophylla. For markers identification, it was performed a complete genome re-sequencing (WGS) of the parents and genotyping-by-sequencing (GBS) of the progeny. The mapping methodology was adapted to the data set, which presents a large amount of SNP markers, only containing diallelic information and with variable genotyping error probability. For this, two strategies were proposed: i) use of genome reference information as aid for map construction; ii) adapting the genotyping error probability parameter in the approach implemented in the software OneMap. The map presented a total size of 1471.91 cM and 1512 markers, with a mean distance between them of 1.85 cM. The markers formed 11 linkage groups, which corresponded to chromosomes of the reference genome. On average, 96.8 \% of chromosomes were covered. The obtained map showed recombination rate pattern similar to other maps constructed for eucalyptus. Also, 61 markers located in other scaffolds in the reference genome were grouped with the 11 groups and they may serve to elucidate the assembly of these. Using the proposed strategies, a suitable integrated map was obtained for the present experiment.
  
  Funding: CNPq, Suzano Papel e Celulose \\ 
  \end{abstract}
  \end{document} 