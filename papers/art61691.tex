
\documentclass[twoside]{article}
\usepackage[affil-it]{authblk}
\usepackage{lipsum} % Package to generate dummy text throughout this template
\usepackage{eurosym}
\usepackage[sc]{mathpazo} % Use the Palatino font
\usepackage[T1]{fontenc} % Use 8-bit encoding that has 256 glyphs
\usepackage[utf8]{inputenc}
\linespread{1.05} % Line spacing-Palatino needs more space between lines
\usepackage{microtype} % Slightly tweak font spacing for aesthetics

\usepackage[hmarginratio=1:1,top=32mm,columnsep=20pt]{geometry} % Document margins
\usepackage{multicol} % Used for the two-column layout of the document
\usepackage[hang,small,labelfont=bf,up,textfont=it,up]{caption} % Custom captions under//above floats in tables or figures
\usepackage{booktabs} % Horizontal rules in tables
\usepackage{float} % Required for tables and figures in the multi-column environment-they need to be placed in specific locations with the[H] (e.g. \begin{table}[H])
\usepackage{hyperref} % For hyperlinks in the PDF

\usepackage{lettrine} % The lettrine is the first enlarged letter at the beginning of the text
\usepackage{paralist} % Used for the compactitem environment which makes bullet points with less space between them

\usepackage{abstract} % Allows abstract customization
\renewcommand{\abstractnamefont}{\normalfont\bfseries} 
%\renewcommand{\abstracttextfont}{\normalfont\small\itshape} % Set the abstract itself to small italic text

\usepackage{titlesec} % Allows customization of titles
\renewcommand\thesection{\Roman{section}} % Roman numerals for the sections
\renewcommand\thesubsection{\Roman{subsection}} % Roman numerals for subsections
\titleformat{\section}[block]{\large\scshape\centering}{\thesection.}{1em}{} % Change the look of the section titles
\titleformat{\subsection}[block]{\large}{\thesubsection.}{1em}{} % Change the look of the section titles

\usepackage{fancyhdr} % Headers and footers
\pagestyle{fancy} % All pages have headers and footers
\fancyhead{} % Blank out the default header
\fancyfoot{} % Blank out the default footer
\fancyhead[C]{X-meeting $\bullet$ November 2017 $\bullet$ S\~ao Pedro} % Custom header text
\fancyfoot[RO,LE]{} % Custom footer text

%----------------------------------------------------------------------------------------
% TITLE SECTION
%----------------------------------------------------------------------------------------

\title{\vspace{-15mm}\fontsize{24pt}{10pt}\selectfont\textbf{EVALUATING THE COWPEA DEHYDRATION STRESS TOLERANCE BASED ON INOSITOL AND RAPHINOSIS PATHWAYS}} % Article title

\author{Jo\~ao Pacifico Bezerra Neto$^1$, Flavia Figueira Aburjaile$^1$, Jos\'e Ribamar Costa Ferreira-neto$^1$, Ana Maria Benko-iseppon$^1$, Mg Santos$^2$}

\affil{1 UFPE, CENTER OF BIOLOGICAL SCIENCES, GENETICS DEPT\\ 2 UFPE, BOTANY DEPT, PLANT PHYSIOLOGY LABORATORY\\ }
\vspace{-5mm}
\date{}

%----------------------------------------------------------------------------------------

\begin{document}

\maketitle % Insert title

\thispagestyle{fancy} % All pages have headers and footers

%----------------------------------------------------------------------------------------
% ABSTRACT
%----------------------------------------------------------------------------------------

\begin{abstract}
Plants evolved to survive in environments that often impose adverse conditions, such as abiotic and biotic stresses. They developed several survival mechanisms that enable the detection of environmental changes, as well as induction of specific responses to imposed stress conditions. Cowpea is one of the most important food and forage legumes in north- and northeastern Brazilian regions and its ability to survive under environmental pressure make it an ideal crop model to study the molecular mechanisms of drought tolerance. In this context, the identification and characterization of inositol (Ins) and raphinosis (RFO) pathways genes was carried out for cowpea in their transcriptome by computational methods. The cowpea transcriptome was assembled from 453 milions of reads, resulting in more than 185,000 non-redundant transcripts, which include transcriptional variants, splicing products. Using seed sequences obtained from the Kyoto Encyclopedia of Genes and Genomes (KEGG) Pathway database 1.119 transcripts were obtained, 521 related to inositol pathway and 598 transcripts associated with raphinosis pathway. It was possible identify 31 KO numbers associated with the raffinose pathway, whereas 29 KO numbers were related to the Inositol pathway. Among all 1.119 transcripts, 468 gene ontology terms were obtained (238 for Ins and 230 for RFO), being reallocated with different enzymatic/metabolic activities that its members perform. For the RFO pathway, the most important biological processes comprise the metabolism of carbohydrates, galactose and raffinose, whereas for Ins we found phosphatidylinositol phosphorylation, lipid catabolism and inositol biosynthesis. Our data pointed out the importance of Ins and RFO availability for cowpea under dehydration, where many cellular processes require many members of both pathways, especially plants which use free Ins to synthesize essential compounds, including those involved in hormonal regulation and stress tolerance.

Funding: CAPES, CNPq, FACEPE.
\end{abstract}
\end{document}