
\documentclass[twoside]{article}
\usepackage[affil-it]{authblk}
\usepackage{lipsum} % Package to generate dummy text throughout this template
\usepackage{eurosym}
\usepackage[sc]{mathpazo} % Use the Palatino font
\usepackage[T1]{fontenc} % Use 8-bit encoding that has 256 glyphs
\usepackage[utf8]{inputenc}
\linespread{1.05} % Line spacing-Palatino needs more space between lines
\usepackage{microtype} % Slightly tweak font spacing for aesthetics

\usepackage[hmarginratio=1:1,top=32mm,columnsep=20pt]{geometry} % Document margins
\usepackage{multicol} % Used for the two-column layout of the document
\usepackage[hang,small,labelfont=bf,up,textfont=it,up]{caption} % Custom captions under//above floats in tables or figures
\usepackage{booktabs} % Horizontal rules in tables
\usepackage{float} % Required for tables and figures in the multi-column environment-they need to be placed in specific locations with the[H] (e.g. \begin{table}[H])
\usepackage{hyperref} % For hyperlinks in the PDF

\usepackage{lettrine} % The lettrine is the first enlarged letter at the beginning of the text
\usepackage{paralist} % Used for the compactitem environment which makes bullet points with less space between them

\usepackage{abstract} % Allows abstract customization
\renewcommand{\abstractnamefont}{\normalfont\bfseries} 
%\renewcommand{\abstracttextfont}{\normalfont\small\itshape} % Set the abstract itself to small italic text

\usepackage{titlesec} % Allows customization of titles
\renewcommand\thesection{\Roman{section}} % Roman numerals for the sections
\renewcommand\thesubsection{\Roman{subsection}} % Roman numerals for subsections
\titleformat{\section}[block]{\large\scshape\centering}{\thesection.}{1em}{} % Change the look of the section titles
\titleformat{\subsection}[block]{\large}{\thesubsection.}{1em}{} % Change the look of the section titles

\usepackage{fancyhdr} % Headers and footers
\pagestyle{fancy} % All pages have headers and footers
\fancyhead{} % Blank out the default header
\fancyfoot{} % Blank out the default footer
\fancyhead[C]{X-meeting $\bullet$ November 2017 $\bullet$ S\~ao Pedro} % Custom header text
\fancyfoot[RO,LE]{} % Custom footer text

%----------------------------------------------------------------------------------------
% TITLE SECTION
%----------------------------------------------------------------------------------------

\title{\vspace{-15mm}\fontsize{24pt}{10pt}\selectfont\textbf{Analysis of the role of an RNA binding protein in the control of gene expression in Trypanosoma cruzi epimastigotes}} % Article title

\author{Wanessa Moreira Goes$^1$, Bruna Mattioly Valente$^1$, Edson Oliveira$^1$, Tha\'{\i}s Silva Tavares$^1$, Fabiano Sviatopolk Mirsky Pais$^2$, Caroline Leonel Vasconcelos de Campos$^1$, Santuza Maria Ribeiro Teixeira$^3$}

\affil{1 UFMG\\ 2 CENTRO DE PESQUISA REN\'E RACHOU, FIOCRUZ\\ 3 INSTITUTE OF BIOLOGICAL SCIENCES, UFMG\\ }
\vspace{-5mm}
\date{}

%----------------------------------------------------------------------------------------

\begin{document}

\maketitle % Insert title

\thispagestyle{fancy} % All pages have headers and footers

%----------------------------------------------------------------------------------------
% ABSTRACT
%----------------------------------------------------------------------------------------

\begin{abstract}
Trypanosoma cruzi, the etiological agent of Chagas disease is a protozoan that has three developmental forms, which are biochemically and morphologically distinct and programed to rapidly respond to the drastic environmental changes this parasite faces during its life cycle. Unlike other eukaryotes, protein-coding genes in this protozoan are transcribed into polycistronic pre-mRNAs that are processed into mature mRNAs through coupled ``trans-splicing'' and poly-adenylation reactions. Because of this, control of gene expression relies mainly on post-transcriptional mechanisms that are mediated by RNA binding proteins (RBP) that control steady-state levels and translation rates of mRNAs. We analysed all sequences corresponding to RNA binding motifs by extracting from Pfam database and using these sequences in BLAST searches against all T. cruzi CL Brener proteins. BLAST hits having E values <10-9 and identity = 85\% identified 253 sequences in the T. cruzi genome containing RNA recognition motif (RRM), PABP, Alba, Pumillio and Zinc Finger motifs. Using RNA-seq data generated from cDNA libraries constructed with mRNA isolated from epimastigotes, trypomastigotes and amastigotes, we analyzed the expression throughout the T. cruzi life cycle of all sequences containing these RNA binding motifs. Among the genes that are up-regulated in epimastigotes, we identified TcCLB.506739.99, which encodes a RBP containing a zinc finger motif, named TcRBP99. A role of this protein related to parasite differentiation was revealed by the characterization of epimastigotes in which this gene was knocked-out: compared to wild type (WT) epimastigotes, TcRBP99 null mutant showed growth inhibition and reduced capacity to differentiate into metacyclic trypomastigotes. RNA-seq analyses comparing total gene expression of wild type epimastigotes and epimastigotes from two knockout cell lines were performed using a workflow that included mapping of reads to a reference genome using STAR, TopHat2 and Bowtie2 tools and differential gene expression (DGE) analyses were performed with Edge R, limma and Deseq2 packages, having padj < 0.05 and log2FoldChange > 1 as cut-off. Our results revealed 12 genes that showed reduced expression in TcRBP99 knockout cell lines compared to WT. One of them encodes a protein annotated as protein associated with differentiation, whose mRNA is up-regulated in wild type epimastigotes compared to other stages. Immunoprecipitation assays showed that TcRBP99 binds to this mRNA, further suggesting a role of TcRBP99 in controlling the expression of proteins that participate in the epimastigote-trypomastigote differentiation.

Funding: CNPq, FAPEMIG, INCTV
\end{abstract}
\end{document}