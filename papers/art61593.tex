
  \documentclass[twoside]{article}
  \usepackage[affil-it]{authblk}
  \usepackage{lipsum} % Package to generate dummy text throughout this template
  \usepackage{eurosym}
  \usepackage[sc]{mathpazo} % Use the Palatino font
  \usepackage[T1]{fontenc} % Use 8-bit encoding that has 256 glyphs
  \usepackage[utf8]{inputenc}
  \linespread{1.05} % Line spacing-Palatino needs more space between lines
  \usepackage{microtype} % Slightly tweak font spacing for aesthetics\[IndentingNewLine]
  \usepackage[hmarginratio=1:1,top=32mm,columnsep=20pt]{geometry} % Document margins
  \usepackage{multicol} % Used for the two-column layout of the document
  \usepackage[hang,small,labelfont=bf,up,textfont=it,up]{caption} % Custom captions under//above floats in tables or figures
  \usepackage{booktabs} % Horizontal rules in tables
  \usepackage{float} % Required for tables and figures in the multi-column environment-they need to be placed in specific locations with the[H] (e.g. \begin{table}[H])
  \usepackage{hyperref} % For hyperlinks in the PDF
  \usepackage{lettrine} % The lettrine is the first enlarged letter at the beginning of the text
  \usepackage{paralist} % Used for the compactitem environment which makes bullet points with less space between them
  \usepackage{abstract} % Allows abstract customization
  \renewcommand{\abstractnamefont}{\normalfont\bfseries} 
  %\renewcommand{\abstracttextfont}{\normalfont\small\itshape} % Set the abstract itself to small italic text\[IndentingNewLine]
  \usepackage{titlesec} % Allows customization of titles
  \renewcommand\thesection{\Roman{section}} % Roman numerals for the sections
  \renewcommand\thesubsection{\Roman{subsection}} % Roman numerals for subsections
  \titleformat{\section}[block]{\large\scshape\centering}{\thesection.}{1em}{} % Change the look of the section titles
  \titleformat{\subsection}[block]{\large}{\thesubsection.}{1em}{} % Change the look of the section titles
  \usepackage{fancyhdr} % Headers and footers
  \pagestyle{fancy} % All pages have headers and footers
  \fancyhead{} % Blank out the default header
  \fancyfoot{} % Blank out the default footer
  \fancyhead[C]{X-meeting $\bullet$ November 2017 $\bullet$ S\~ao Pedro} % Custom header text
  \fancyfoot[RO,LE]{} % Custom footer text
  %----------------------------------------------------------------------------------------
  % TITLE SECTION
  %---------------------------------------------------------------------------------------- 
 
 \title{\vspace{-15mm}\fontsize{24pt}{10pt}\selectfont\textbf{ Evaluation of differentially expressed proteins during Leishmania major infection in murine macrophages lacking nitric oxide synthase }} % Article title
  
  
  \author{ Victor Hugo Toledo$^{1}$, Djalma de Souza Lima Junior$^{1}$, Lívia Rosa Fernandes$^{2}$, Giuseppe Palmisano$^{2}$, Luiza A. Castro-Jorge$^{1}$, Dario Simões Zamboni$^{1}$, }
  
  \affil{ 1 Faculdade de Medicina de Ribeirão Preto - Universidade de São Paulo

2 Instituto de Ciências Biomédicas - Universidade de São Paulo

 }
  \vspace{-5mm}
  \date{}
  
  %---------------------------------------------------------------------------------------- 
  
  \begin{document}
  
  
  \maketitle % Insert title
  
  
  \thispagestyle{fancy} % All pages have headers and footers
  %----------------------------------------------------------------------------------------  
  % ABSTRACT
  
  %----------------------------------------------------------------------------------------  
  
  \begin{abstract}
  Leishmaniasis is a neglected tropical disease that can have 3 different presentations, cutaneous, mucocutaneous, or visceral leishmaniasis. It is caused by a diverse group of protozoan parasites, Leishmania, and is transmitted by certain types of sandflies. It is estimated that 1.5 million people are infected each year in more than 98 countries where the disease is endemic. Until now, vaccination and drug therapy have failed to control the disease. The main mechanisms responsible for controlling Leishmania replication involves the production of nitric oxide (NO), generated by inducible NO synthase (iNOS) following activation by IFN? and also reactive oxygen species (ROS), generated by the respiratory burst. Hence, studies evaluating changes in protein expression after Leishmania major infection in wild type macrophages, and macrophages lacking or superexpressing iNOS could help in the discovery of novel targets for the control of L. major. In order to identify proteins differentially expressed (DEPs) related to these functions, we analyzed protein extracts from C57BL/6J bone marrow derived macrophages (BMDMs) infected or not with Leishmania major and iNOS-/- BMDMs, using mass spectometry. Differential regulated proteins were selected based on several statistical analyses (t-test, LIMMA, ROTS and SAM), performed using RStudio and relevant packages, as to combine results from several sources and to choose the most suitable method to improve the confidence. DEPs were then submitted to biological network analyses using Enrichment Map and Gene Ontology related tools (such as g:Profiler and DAVID)  to define enriched functionally related genes. In addition, we evaluated protein-protein physical and functional interactions with STRING database and also pathway abundances through Ingenuity Pathway Knowledge Base to improve these results. Thereby, we identified proteins differentially modulated during L. major infection course, which allowed us to define important altered biological processes, such as early endosome to late endosome transport. We expect our results to widen the understanding of the infection control and to unravel new information for further studies.
  
  Funding: Nenhum \\ 
  \end{abstract}
  \end{document} 