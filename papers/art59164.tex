
\documentclass[twoside]{article}
\usepackage[affil-it]{authblk}
\usepackage{lipsum} % Package to generate dummy text throughout this template
\usepackage{eurosym}
\usepackage[sc]{mathpazo} % Use the Palatino font
\usepackage[T1]{fontenc} % Use 8-bit encoding that has 256 glyphs
\usepackage[utf8]{inputenc}
\linespread{1.05} % Line spacing-Palatino needs more space between lines
\usepackage{microtype} % Slightly tweak font spacing for aesthetics

\usepackage[hmarginratio=1:1,top=32mm,columnsep=20pt]{geometry} % Document margins
\usepackage{multicol} % Used for the two-column layout of the document
\usepackage[hang,small,labelfont=bf,up,textfont=it,up]{caption} % Custom captions under//above floats in tables or figures
\usepackage{booktabs} % Horizontal rules in tables
\usepackage{float} % Required for tables and figures in the multi-column environment-they need to be placed in specific locations with the[H] (e.g. \begin{table}[H])
\usepackage{hyperref} % For hyperlinks in the PDF

\usepackage{lettrine} % The lettrine is the first enlarged letter at the beginning of the text
\usepackage{paralist} % Used for the compactitem environment which makes bullet points with less space between them

\usepackage{abstract} % Allows abstract customization
\renewcommand{\abstractnamefont}{\normalfont\bfseries} 
%\renewcommand{\abstracttextfont}{\normalfont\small\itshape} % Set the abstract itself to small italic text

\usepackage{titlesec} % Allows customization of titles
\renewcommand\thesection{\Roman{section}} % Roman numerals for the sections
\renewcommand\thesubsection{\Roman{subsection}} % Roman numerals for subsections
\titleformat{\section}[block]{\large\scshape\centering}{\thesection.}{1em}{} % Change the look of the section titles
\titleformat{\subsection}[block]{\large}{\thesubsection.}{1em}{} % Change the look of the section titles

\usepackage{fancyhdr} % Headers and footers
\pagestyle{fancy} % All pages have headers and footers
\fancyhead{} % Blank out the default header
\fancyfoot{} % Blank out the default footer
\fancyhead[C]{X-meeting $\bullet$ November 2017 $\bullet$ S\~ao Pedro} % Custom header text
\fancyfoot[RO,LE]{} % Custom footer text

%----------------------------------------------------------------------------------------
% TITLE SECTION
%----------------------------------------------------------------------------------------

\title{\vspace{-15mm}\fontsize{24pt}{10pt}\selectfont\textbf{CINDEX: a software for protein ranking through network modeling based on graph theory}} % Article title

\author{James Shiniti Nagai$^1$, Hugo Rody Vianna Silva$^1$, Alexandre Hild Aono$^1$, Estela Araujo Costa$^1$, Reginaldo Massanobu Kuroshu$^1$}

\affil{1 UNIFESP\\ }
\vspace{-5mm}
\date{}

%----------------------------------------------------------------------------------------

\begin{document}

\maketitle % Insert title

\thispagestyle{fancy} % All pages have headers and footers

%----------------------------------------------------------------------------------------
% ABSTRACT
%----------------------------------------------------------------------------------------

\begin{abstract}
Metabolic networks have increased in complexity throughout the evolution of species becoming strongly connected metabolic blocks, which seems to have given stability to the flux of information in these networks and, possibly, turning organisms into more relaxed ones to adapt. We present the first version of CINDEX, a software for protein ranking through the modeling of protein-protein interaction (PPI) networks based on graph theory; it can be downloaded at https://github.com/hugorody/cindex. The tool accepts as input an organism's subset of proteins provided by the user and uses PPI information from the KEGG Pathway database to model a specific metabolic network throughout a directed graph. The proteins are set as the nodes of the graph, whereas the connections among the nodes are given by arcs (directed edges) that are created when the product of a protein is a substrate to another. CINDEX then calculates the centrality degree of nodes using three different metrics - Degree Centrality, Closeness Centrality, and Betweenness Centrality -, which provides to the user different biological perspectives for protein (node) ranking. Additionally, our software searches for lethal bottleneck proteins - proteins represented by nodes that disconnect the network when removed, thus essential to keep the flux of information in a network and whose inactivation could be lethal to the organism. Finally, the clustering coefficient is calculated to indicate the presence of protein clusters within the networks; a subset of interacting proteins likely to control many cellular processes.

Funding: CAPES
\end{abstract}
\end{document}