
  \documentclass[twoside]{article}
  \usepackage[affil-it]{authblk}
  \usepackage{lipsum} % Package to generate dummy text throughout this template
  \usepackage{eurosym}
  \usepackage[sc]{mathpazo} % Use the Palatino font
  \usepackage[T1]{fontenc} % Use 8-bit encoding that has 256 glyphs
  \usepackage[utf8]{inputenc}
  \linespread{1.05} % Line spacing-Palatino needs more space between lines
  \usepackage{microtype} % Slightly tweak font spacing for aesthetics\[IndentingNewLine]
  \usepackage[hmarginratio=1:1,top=32mm,columnsep=20pt]{geometry} % Document margins
  \usepackage{multicol} % Used for the two-column layout of the document
  \usepackage[hang,small,labelfont=bf,up,textfont=it,up]{caption} % Custom captions under//above floats in tables or figures
  \usepackage{booktabs} % Horizontal rules in tables
  \usepackage{float} % Required for tables and figures in the multi-column environment-they need to be placed in specific locations with the[H] (e.g. \begin{table}[H])
  \usepackage{hyperref} % For hyperlinks in the PDF
  \usepackage{lettrine} % The lettrine is the first enlarged letter at the beginning of the text
  \usepackage{paralist} % Used for the compactitem environment which makes bullet points with less space between them
  \usepackage{abstract} % Allows abstract customization
  \renewcommand{\abstractnamefont}{\normalfont\bfseries} 
  %\renewcommand{\abstracttextfont}{\normalfont\small\itshape} % Set the abstract itself to small italic text\[IndentingNewLine]
  \usepackage{titlesec} % Allows customization of titles
  \renewcommand\thesection{\Roman{section}} % Roman numerals for the sections
  \renewcommand\thesubsection{\Roman{subsection}} % Roman numerals for subsections
  \titleformat{\section}[block]{\large\scshape\centering}{\thesection.}{1em}{} % Change the look of the section titles
  \titleformat{\subsection}[block]{\large}{\thesubsection.}{1em}{} % Change the look of the section titles
  \usepackage{fancyhdr} % Headers and footers
  \pagestyle{fancy} % All pages have headers and footers
  \fancyhead{} % Blank out the default header
  \fancyfoot{} % Blank out the default footer
  \fancyhead[C]{X-meeting $\bullet$ November 2017 $\bullet$ S\~ao Pedro} % Custom header text
  \fancyfoot[RO,LE]{} % Custom footer text
  %----------------------------------------------------------------------------------------
  % TITLE SECTION
  %---------------------------------------------------------------------------------------- 
 
 \title{\vspace{-15mm}\fontsize{24pt}{10pt}\selectfont\textbf{ TPP riboswitch analysis using molecular dynamic with different force fields }} % Article title
  
  
  \author{ Rodrigo Bentes Kato$^{1}$, Jadson Claudio Belchior$^{1}$, Debora Antunes$^{2}$, }
  
  \affil{ 1 UFMG

2 Instituição: Instituto Oswaldo Cruz - FIOCRUZ

 }
  \vspace{-5mm}
  \date{}
  
  %---------------------------------------------------------------------------------------- 
  
  \begin{document}
  
  
  \maketitle % Insert title
  
  
  \thispagestyle{fancy} % All pages have headers and footers
  %----------------------------------------------------------------------------------------  
  % ABSTRACT
  
  %----------------------------------------------------------------------------------------  
  
  \begin{abstract}
  Riboswitch RNAs are important in bacterial metabolism and represent a promising class of antibiotic targets for treatment of infectious disease. Molecular dynamics simulations are used for interpreting experimental data and to predict new experiments. The present work is concerned to analyze the parametrizations most used in literature to describe force fields. A comparison is done between Charmm27 force field and others such as Amber99, AmberGS and Amber2014. Applications are carried out for tackling RNA molecules. The Gromacs (GROningen MAchine for Chemical Simulations) software are used to analyze RNAs structure and dynamic under a broad variety of patterns. In particular, Thiamin pyrophosphate (TPP) riboswitch was considered to evaluate strategies for studying parametrization of force fields. As it is well-known this RNA is important for regularization of gene expression through a variety of mechanisms in archaea, bacteria and eukaryotes. Main preliminary results are concerned with the preparation of RNA (2gdi.pdb) using a box of 7x4x3 angstrom, solvated with water TIP3 and neutralized with magnesium (Mg). The dynamics were carried out considering at this preliminary analysis up to 200 ns and the final analyzes were done using rmsd (root mean square deviation) for the trajectories. At this stage the outcome results demonstrated that AmberGS stabilized the RNA with the lowest rmsd, but closer to Amber99. In general, comparison using rmsd against others force fields showed: AmberGS 0.2182, Amber99 0.2312, Charmm27 0.5325 and Amber2014 0.6469. Therefore, AmberGS and Amber99 produced the best force fields for describing this particular RNA and it might be a good estimation to similar analysis for others systems and this is under investigation.
  
  Funding: FAPEMIG, CAPES, CNPq \\ 
  \end{abstract}
  \end{document} 