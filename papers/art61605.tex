
\documentclass[twoside]{article}
\usepackage[affil-it]{authblk}
\usepackage{lipsum} % Package to generate dummy text throughout this template
\usepackage{eurosym}
\usepackage[sc]{mathpazo} % Use the Palatino font
\usepackage[T1]{fontenc} % Use 8-bit encoding that has 256 glyphs
\usepackage[utf8]{inputenc}
\linespread{1.05} % Line spacing-Palatino needs more space between lines
\usepackage{microtype} % Slightly tweak font spacing for aesthetics

\usepackage[hmarginratio=1:1,top=32mm,columnsep=20pt]{geometry} % Document margins
\usepackage{multicol} % Used for the two-column layout of the document
\usepackage[hang,small,labelfont=bf,up,textfont=it,up]{caption} % Custom captions under//above floats in tables or figures
\usepackage{booktabs} % Horizontal rules in tables
\usepackage{float} % Required for tables and figures in the multi-column environment-they need to be placed in specific locations with the[H] (e.g. \begin{table}[H])
\usepackage{hyperref} % For hyperlinks in the PDF

\usepackage{lettrine} % The lettrine is the first enlarged letter at the beginning of the text
\usepackage{paralist} % Used for the compactitem environment which makes bullet points with less space between them

\usepackage{abstract} % Allows abstract customization
\renewcommand{\abstractnamefont}{\normalfont\bfseries} 
%\renewcommand{\abstracttextfont}{\normalfont\small\itshape} % Set the abstract itself to small italic text

\usepackage{titlesec} % Allows customization of titles
\renewcommand\thesection{\Roman{section}} % Roman numerals for the sections
\renewcommand\thesubsection{\Roman{subsection}} % Roman numerals for subsections
\titleformat{\section}[block]{\large\scshape\centering}{\thesection.}{1em}{} % Change the look of the section titles
\titleformat{\subsection}[block]{\large}{\thesubsection.}{1em}{} % Change the look of the section titles

\usepackage{fancyhdr} % Headers and footers
\pagestyle{fancy} % All pages have headers and footers
\fancyhead{} % Blank out the default header
\fancyfoot{} % Blank out the default footer
\fancyhead[C]{X-meeting $\bullet$ November 2017 $\bullet$ S\~ao Pedro} % Custom header text
\fancyfoot[RO,LE]{} % Custom footer text

%----------------------------------------------------------------------------------------
% TITLE SECTION
%----------------------------------------------------------------------------------------

\title{\vspace{-15mm}\fontsize{24pt}{10pt}\selectfont\textbf{Genes and pathways modulated during Guillain-Barr\'e Syndrome}} % Article title

\author{Raulzito Fernandes Moreira$^1$, Paulo Ricardo Porf\'{\i}rio do Nascimento$^2$, Gl\'oria Regina de G\'ois Monteiro$^3$, Mario Emilio Teixeira Dourado Junior$^3$, Selma Maria Bezerra Jeronimo$^3$, Jo\~ao Paulo Matos Santos Lima$^4$}

\affil{1 PROGRAMA DE P\'OS-GRADUA\c{C}\~AO EM BIOTECNOLOGIA DOS RECURSOS NATURAIS, UNIVERSIDADE FEDERAL DO CEAR\'A\\ 2 INSTITUTO DE MEDICINA TROPICAL DO RIO GRANDE DO NORTE, UFRN.\\ 3 INSTITUTO DE MEDICINA TROPICAL DO RIO GRANDE DO NORTE, UFRN\\ 4 PROGRAMA DE P\'OS-GRADUA\c{C}\~AO EM BIOINFORM\'ATICA, UFRN\\ }
\vspace{-5mm}
\date{}

%----------------------------------------------------------------------------------------

\begin{document}

\maketitle % Insert title

\thispagestyle{fancy} % All pages have headers and footers

%----------------------------------------------------------------------------------------
% ABSTRACT
%----------------------------------------------------------------------------------------

\begin{abstract}
Guillain-Barr\'e syndrome (GBS) is an acute polyradiculoneuropathy, monophasic, and since the eradication of poliomyelitis is the principal cause of paralysis in the world. This syndrome seems to have an autoimmune component characterized, in part, by molecular mimicry with production of antibodies that causes severe damage to peripheral nerves. Such damage results in symptoms, which include acute flaccid paralysis. About 30\% of the cases need respiratory assistance. GBS cases is usually preceded by infection agents, such  as Campylobacter jejuni and viral infections. It seems that the pathogenesis of C. jejuni in GBS is associated with anti-gangliosides antibodies, which cross react with gangliosides present in the nerve axolemma, mainly in the peripheral nerves. Hence, the present study aimed to analyze transcriptomic libraries from patients with GBS, diagnosed with the different subtypes (demyelinating, axonal and Miller-Fisher), in order to identify genes and key pathways related to GBS development and potential target for modulation. For this, 24 libraries were obtained from 12 Brazilian patients diagnosed with GBS (6 from demyelinating subtype, 4 from axonal form and 2 from Miller-Fisher form), in two distinct phases, symptomatic/acute phase and after complete recovery. The quality analysis, alignment, assembly and global gene expression were performed using FastQC, Bowtie, TopHat, Cufflinks (Cuffmerge) HTSeq and edger. Approximately 2000 genes were differentially expressed between symptomatic and the recovery phase (p<0.01 and log fold change 1.5). Transcript annotation based on GO and KEGG terms showed changes in expression of genes related to inflammation, as TNF signaling pathway, toll-like and NOD-like receptor signaling pathways. Also, pathways related to neurodegenerative, autoimmune and infectious diseases were enriched during symptomatic phase when compared to recovery phase. These results are in accordance to other previous studies and provide an overview of possible responses during the course of GBS.

Funding: CNPq, NIH, FUNCAP, CAPES.
\end{abstract}
\end{document}