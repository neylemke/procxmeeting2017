
\documentclass[twoside]{article}
\usepackage[affil-it]{authblk}
\usepackage{lipsum} % Package to generate dummy text throughout this template
\usepackage{eurosym}
\usepackage[sc]{mathpazo} % Use the Palatino font
\usepackage[T1]{fontenc} % Use 8-bit encoding that has 256 glyphs
\usepackage[utf8]{inputenc}
\linespread{1.05} % Line spacing-Palatino needs more space between lines
\usepackage{microtype} % Slightly tweak font spacing for aesthetics

\usepackage[hmarginratio=1:1,top=32mm,columnsep=20pt]{geometry} % Document margins
\usepackage{multicol} % Used for the two-column layout of the document
\usepackage[hang,small,labelfont=bf,up,textfont=it,up]{caption} % Custom captions under//above floats in tables or figures
\usepackage{booktabs} % Horizontal rules in tables
\usepackage{float} % Required for tables and figures in the multi-column environment-they need to be placed in specific locations with the[H] (e.g. \begin{table}[H])
\usepackage{hyperref} % For hyperlinks in the PDF

\usepackage{lettrine} % The lettrine is the first enlarged letter at the beginning of the text
\usepackage{paralist} % Used for the compactitem environment which makes bullet points with less space between them

\usepackage{abstract} % Allows abstract customization
\renewcommand{\abstractnamefont}{\normalfont\bfseries} 
%\renewcommand{\abstracttextfont}{\normalfont\small\itshape} % Set the abstract itself to small italic text

\usepackage{titlesec} % Allows customization of titles
\renewcommand\thesection{\Roman{section}} % Roman numerals for the sections
\renewcommand\thesubsection{\Roman{subsection}} % Roman numerals for subsections
\titleformat{\section}[block]{\large\scshape\centering}{\thesection.}{1em}{} % Change the look of the section titles
\titleformat{\subsection}[block]{\large}{\thesubsection.}{1em}{} % Change the look of the section titles

\usepackage{fancyhdr} % Headers and footers
\pagestyle{fancy} % All pages have headers and footers
\fancyhead{} % Blank out the default header
\fancyfoot{} % Blank out the default footer
\fancyhead[C]{X-meeting $\bullet$ November 2017 $\bullet$ S\~ao Pedro} % Custom header text
\fancyfoot[RO,LE]{} % Custom footer text

%----------------------------------------------------------------------------------------
% TITLE SECTION
%----------------------------------------------------------------------------------------

\title{\vspace{-15mm}\fontsize{24pt}{10pt}\selectfont\textbf{CeTICSdb Database resources and functionalities for the integration of -omics data and mathematical models of signaling networks}} % Article title

\author{Milton Y. Nishiyama-jr$^1$, Marcelo S. Reis$^1$, Bruno Ferreira de Souza$^2$, Henrique Cursino Vieira$^3$, Daniel F. Silva$^4$, In\'acio L.m. Junqueira-de-azevedo$^5$, Julia P.c. da Cunha$^1$, Junior Barrera$^6$, Leo K. Iwai$^7$, Solange M.t. Serrano$^8$, Hugo A. Armelin$^1$}

\affil{1 INSTITUTO BUTANTAN\\ 2 ECC-CETICS, INSTITUTO BUTANTAN\\ 3 LECC-CETICS, INSTITUTO BUTANTAN\\ 4 ESCOLA POLIT\'ECNICA, USP S\~AO PAULO\\ 5 LETA-CETICS, INSTITUTO BUTANTAN, S\~AO PAULO, BRAZIL\\ 6 INSTITUTO DE MATEM\'ATICA E ESTAT\'ISTICA, USP\\ 7 LETA-CETICS, INSTITUTO BUTANTAN\\ 8 LETA-CETICS, INSTITUTO BUTANTAN, S\~AO PAULO\\ }
\vspace{-5mm}
\date{}

%----------------------------------------------------------------------------------------

\begin{document}

\maketitle % Insert title

\thispagestyle{fancy} % All pages have headers and footers

%----------------------------------------------------------------------------------------
% ABSTRACT
%----------------------------------------------------------------------------------------

\begin{abstract}
The understanding of biological systems and signaling networks processes constitutes not only a conceptual challenge but a multi-factorial problem if based on different experimental conditions, treatments, time points, etc. The Center of Toxins, Immune-response and Cell Signaling (CeTICS) aims to understand the behavior of biological systems in specific treatments and conditions, using the -omics data and signaling networks analysis; The studies and research in CeTICS project are intrinsically interdisciplinary, which is coupled to the -omics data and heterogeneous knowledge and implies a necessity of data organization and integration to carry out scientific investigations for the generation of new insights and meaningful results. The CeTICSdb aims to provide a dynamic, user-friendly integrated system, for fully support research management, data management, perform customized on the fly analysis, simulations and apply pattern recognition methods for integration of multiple -omics data.CeTICSdb is the core of ARTISiN, an amalgam of repositories and tools, both public and in-house built ones for analysis of signaling networks. ARTISiN will allow a communication between CeTICSdb and SigNetSim, a tool for generation of dynamical models. Moreover, it has been designed for the integration of multi-omic data and mathematical modeling of signaling network. The CeTICSdb has been built with Django (Python web framework) and is composed of multiple components, which will allow to efficiently evolving it into a data management framework, requiring fewer manual changes, especially in the development of new applications. The platform will integrate the data between multiple platforms such as Galaxy and GBrowse, and public components such as Biomodels database, Cytoscape plugin and Mascot. To evaluate and test the platform, we integrated transcriptome and protein expression profiles with Metabolic Pathways to: i) estimate the pathways relative abundance between different conditions; ii) define and compare the functional activity for the pathways in each condition; iii) infer networks based on STRING information. 
Finally, our mid-term objective is to make the CeTICSdb platform available as a dry lab to the scientific community and the core for ARTISiN. It has already been used as reference in the projects Biota/FAPESP and Tityus Scorpions species. CeTICSdb is free software licensed under GNU Affero General Public License (AGPL) and is available at http://cetics.butantan.gov.br/ceticsdb.

Funding: CNPq, FAPESP (2011/22619-7,  2012/00177-5,  2013/07467-1)
\end{abstract}
\end{document}